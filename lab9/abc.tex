\documentclass[11pt]{book}
\usepackage[T1]{fontenc}
\title{Historyczny Drakula?}
\author{Antoni Diller}
\begin{document}
\maketitle
\noindent
Liczne przes\l anki \'swiadcz\k{a} o tym, \.ze ksi\k{a}\.z\k{e}
Drakula Brama Stokera mia\l{} pierwowz\'or w historycznej
\begin{equation}
A\cup B = \{x \colon (x \in A) \vee (x \in B)\}
\end{equation}
postaci W\l{}ada III \c{T}epe\c{s}a (1431--1476), hospodara
Wo\l{}oszczyzny, krainy w po\l{}udniowo-wschodniej
Transylwanii. Ksi\k{a}\.z\k{e} \'ow jest le\emph\textbf{HSIFGFYEFGY}piej znany jako
\begin{center}
W\l{}ad Palownik z powodu zami\l{}owania do osobliwie
przykrego sposobu wymierzania kar, kt\'orego szczeg\'o\l{}y
\end{center}
lepiej pozostawi\'c wyobra\'zni czytelnika. Obecnie jest on
czczony jako narodowy bohater Rumunii.
\begin{description}
\item Monday -- Moon's day,
\item Tuesday -- Tiu's day,
\item Wednesday -- Woden's day,
\item Thursday -- Thor's day
\item Friday -- Freya's day,
\item Saturday -- Saturn's day,
\item Sunday -- Sun's.
\end{description}
\chapter{sfihdrfguihgidrugh}
qwfjfueifhgefyifgywefgfuygyfguyfgdhfsfgfefgfgvefgvfvb
fsdfsffd\begin{quote} fsffsf \end{quote}
\end{document}